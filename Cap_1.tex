\chapter{Introducción}

El área de Aprendizaje Automático es muy fructífera y cuenta con abundante investigación. Uno de los objetivos de esta área es resolver problemas de clasificación, es decir, dadas un conjunto de características observadas, asignar una categoría basado en ejemplos vistos anteriormente. Estos algoritmos clasificadores son entrenados mediante algoritmos de aprendizaje supervisados.

Un problema de clasificación que tiene solamente dos categorías es llamado binario. Una de las aplicaciones en el mundo real de estos clasificadores binarios se halla en el área de calificación crediticia, donde la tarea es clasificar a las personas en buenos y malos pagadores, basandose en el historial de crédito y otras variables de comportamiento que cada entidad financiera pueda recolectar.

Al utilizar los algoritmos de aprendizaje automático en problemas de calificación crediticia, se habilita una poderosa herramienta que podría mejorar las tasas de interés a las que son ofrecidas los préstamos e incrementar la inclusión financiera a nivel global.

\section{Motivación y Contexto}

El momento en que se publica este trabajo es muy emocionante, puesto que en el entorno emprendedor peruano han surgido varias \textit{Fintech} (entidad que brinda servicios financieros basados en innovación tecnológica) que aprovechan el potencial del aprendizaje automático para expandir el porcentaje de la población que tiene acceso al crédito, que actualmente está rondando el 30\% \citep{carballo2019fintech}.

Las instituciones financieras tradicionales se han dado cuenta de esto y también están empezando a utilizar herramientas de aprendizaje automático para mantenerse competitivas.

Desde el lado de la ciencia, este florecimiento del campo de aplicación ha impulsado la investigación en esta área, de modo que ahora se tienen nuevos algoritmos como \ac{XGBoost} \citep{Chen:2016:XST:2939672.2939785} y nuevas técnicas de \textit{bagging} (entrenamiento de varias instancias del modelo con diferentes subconjuntos de la base de datos) \citep{breiman1996bagging}.

Sin embargo, cuando se aplican estas técnicas, muchas veces se olvidan características intrínsecas de los datos en esta área, lo que lleva a crear modelos subóptimos. De ahí surge la necesidad de enfocarnos en estas características particulares para mejorar el estado del arte existente.

\section{Planteamiento del Problema}

% Donde y cuando aparece el problema? A quien afecta?
Al entrenar modelos para predecir morosidad, los conjuntos de datos siempre son desbalanceados. Esto ocurre en todas las entidades financieras del mundo. El desbalance puede encontrarse desde un moderado 30\% hasta llegar a menos de 1\% de instancias de la clase minoritaria, que suelen ser los créditos morosos.

% Qué intentos ha habido de solucionar el problema?
Se ha intentado solucionar este problema haciendo sobremuestreo y submuestro \citep{drummond2003c4}. También se ha intentado ajustar los parámetros de los modelos para compensar el desbalance. Pero estos métodos aún no han tenido un efecto significativo en el desempeño de los modelos.

% Qué pasa si no se soluciona el problema?
Los modelos de crédito ineficientes perjudican a las entidades financieras y los prestatarios. Las entidades financieras tienen elevadas tasas de morosidad y costos elevados de adquisión de clientes. Mientras que los prestatarios encuentran más caro y mas difícil acceder a un crédito \citep{avery2009credit}.

% El problema tiene implicaciones más amplias?
% Cuál es la relevancia cokmputacional
Resolver este problema podría mejorar el acceso al crédito para millones de personas, especialmente en latinoamérica, donde hay países como Perú, con el 70\% de la población que aún no está bancarizada. Por otro lado, desde el punto de vista computacional, permitiría abrir nuevas áreas de investigación para las técnicas de aprendizaje de máquina para conjuntos de datos desbalanceados.


\section{Objetivos}

Aplicar en el área de calificación crediticia, un reciente método de ensamble para clasificación de datos desbalanceados \citep{sun2015novel}. Y compararlo con otros modelos del estado del arte.


\subsection{Objetivos Específicos}

\begin{itemize}
	\item Implementar el método de ensamble para poder realizar pruebas con él.

	\item Elegir 3 bases de datos para ejecutar los experimentos.

	\item Comparar el desempeño de cuatro clasificadores: (\ac{LR}, \ac{MLP}, \ac{SVM} y \ac{AD}); contra su versión ensamblada. Para comprobar si el ensamble es mejor que los clasificadores base.

	\item Comparar el desempeño de los cuatro clasificadores ensamblados en el punto anterior contra otros algoritmos de ensamble del estado del arte: \ac{RF} y \ac{XGBoost}. Para comprobar si es un método de ensamble al menos competitivo.

	\item Comparar el desempeño de los cuatro clasificadores ensamblados, con los resultados de otros trabajos de investigación del estado del arte, usando conjuntos de datos de referencia. De esta manera verificar si hay una mejora respecto al estado del arte.
\end{itemize}

\section{Organización de la tesis}

En el capítulo 2 se dará un marco teórico al mismo tiempo que una revisión del Estado del Arte. Se abordarán temas como la historia de la minería de datos y los puntajes de crédito, estudios comparativos y problemas identificados con la investigación actual. También se explicarán los algoritmos de clasificación usados en esta investigación, los algoritmos de ensamble referenciales y los algoritmos de evaluación. Finalmente se hará un estudio del trabajo relacionado.

En el capítulo 3 se explicará la propuesta: el algoritmo de ensamble para datos desbalanceados. Se detallarán los conjuntos de datos que serán utilizados y la metodología para realizar los experimentos.

Luego en el capítulo 4 se presentarán los experimentos en detalle y sus resultados en forma de tabla, al mismo tiempo, se harán pruebas de significancia estadística y observaciones interesantes.

Finalmente en el capítulo 5 se bosquejan las conclusiones, problemas encontrados, recomendaciones y se proponen trabajos futuros.

%\section{Cronograma}

%Esta sección sólo es para aquellos alumnos que estén presentando su plan de tesis. Esta sección no va en la tesis final.

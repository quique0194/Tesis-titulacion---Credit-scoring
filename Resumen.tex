\begin{resumen}
La investigación actual en calificación crediticia no ha prestado atención al desbalanceo presente en los conjuntos de datos. Por esta razón, en este trabajo se importó un método reciente de ensamble para clasificar datos desbalanceados en el dominio de crédito. Se usaron cuatro clasificadores base de diversas familias, y tres conjuntos de datos heterogéneos como prueba. Luego de ejecutar los experimentos, los resultados son bastante alentadores; el área bajo la curva (AUC) mejoró en diez de doce clasificadores base. Además, los clasificadores ensamblados creados son estadísticamente superiores a los algoritmos del estado del arte \textit{Random Forest} y \textit{XGBoost}. Finalmente, al comparar los resultados de este estudio con los resultados de otros estudios en dos conjuntos de datos de referencia se ratifica que los resultados son bastante competitivos.

\textbf{Palabras Clave:} Calificación crediticia, Desbalanceo de datos, Aprendizaje de máquina, Ensamble de modelos de clasificación
\end{resumen}



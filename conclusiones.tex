\chapter{Conclusiones y Trabajos Futuros}\label{chap:conclusiones}

%En primer lugar debes escribir las conclusiones generales de tu trabajo. evita escribirlas en forma de viñetas. Simplemente utiliza texto continuo.

En esta tesis se implementó un reciente método de ensamble para clasificación de datos desbalanceados, y se comparó su desempeño con otras técnicas de calificación crediticia del estado del arte. Se tomó como base la hipótesis de que no se han explotado las características de los datos crediticios típicos, como el desbalanceo de las instancias.

Se elegieron tres conjuntos de datos para realizar los experimentos, con diferentes características: El conjunto de datos Apurata, inédito de Perú; el conjunto de datos LendingClub, con casi 200 mil observaciones; y el conjunto de datos Alemán que ha sido ampliamente usado en varias investigaciones desde 1994.

Se analizó la mejora de cuatro clasificadores base al aplicarles el método de ensamble propuesto. Los resultados fueron muy positivos. De los doce clasificadores base (cuatro por cada conjunto de datos), diez incrementaron su \ac{AUC} de forma significativa.

Se probó una red neuronal profunda llamada \ac{DBN}. Sus resultados no fueron buenos debido a la complejidad del modelo y el tamaño reducido de los datos. Debido a la complejidad inherente de este clasificador, no se le creó una versión ensamblada.

Se comparó los mejores ensambles propuestos con \ac{RF} y \ac{XGBoost}, dos algoritmos del estado del arte. Los resultados fueron que en dos de los tres conjuntos de datos se consiguió mejorar el desempeño medido en términos de \ac{AUC}.

Finalmente, al comparar los resultados con otras investigaciones referenciales, se pudo comprobar que el desempeño es competitivo dentro del estado del arte. Esto es alentador y motiva a profundizar en la exploración de técnicas que se enfoquen en otras carácterísticas de los conjuntos de datos crediticios, como la no estacionalidad y la sensibilidad al costo.


\section{Problemas encontrados}

El procesamiento de la información es costoso en términos de tiempo del investigador. Hay que entender las variables para poder transformarlas o eliminarlas de forma efectiva.

Hay demasiadas combinaciones de modelos y parámetros, y esto se puede complicar aún más si se agregan métodos de \textit{bagging}, por lo que se requiere muchas ejecuciones de los modelos para encontrar los parámetros mas adecuados. Esto, al igual que el punto anterior es costoso en términos de tiempo del investigador.

Los resultados de una sóla ejecución pueden no ser confiables, esto es fácilmente detectable si al ejectuar varias veces las pruebas los resultados no son estables. Esto es especialmente relevante con pocas instancias, como sucedió en esta investigación.

A pesar de que el mejor método para comparar algoritmos sea el AUC; ya que no depende de la selección de un punto de corte y es insensible al desbalanceo de las clases; otros estudios del estado del arte insisten en utilizar otras métricas sub-óptimas como la exactitud. De modo que uno se ve forzado a usar estas mismas métricas para poder comparar sus resultados con el estado del arte de forma mas o menos efectiva.

% La segunda parte de este capítulo corresponde a los problemas encontrados. esta seccion es muy importante para que los siguientes estudiantes que hagan algo en esta línea no cometan los mismos errores y tu tesis sea un buen peldaño para avanzar más rápido.

\section{Recomendaciones}

Si la limpieza de los datos no es parte de la investigación, hay que procurar tomar conjuntos de datos ya pre-procesados.

Si se tienen pocas instancias y los resultados cambian con cada ejecución de las pruebas, hay que utilizar cross validation con repetición. El número de repeticiones se puede escoger a mano hasta que los resultados sean estables. Basta probar que los resultados sean estables con un modelo simple como \ac{LR} y se puede asumir que van a seguir siendo estables con modelos más complejos.

Usar conjuntos de datos heterogéneos permite obtener conclusiones más generalizables, y hacer una mejor comparación que no esté tan sesgada por características particulares de ciertos datos.

%En esta sección el tesista debe reflejar que la tesis ha permitido adquirir nuevos conocimientos que podrían servir para guiar otros trabajos en el futuro.

\section{Trabajo futuro}

En este trabajo vimos que atender algunas de las características de los conjuntos de datos de crédito ayudan a mejorar el desempeño de los modelos. Por lo tanto los trabajos futuros se pueden enfocar en explotar las características que no fueron cubiertas en esta investigación, como lo son la sensibilidad al costo, y la estacionalidad de los datos.

% En base a los puntos anteriores es recomendable que tu tesis también sugiera trabajos futuros. Esta sección es esecialmente útil para otras ideas de tesis.

%Todo este capítulo no debe ser más de unas 4 páginas.